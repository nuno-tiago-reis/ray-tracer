%%%%%%%%%%%%%%%%%%%%%%%%%%%%%%%%%%%%%%%%%%%%%%%%%%%%%%%%%%%%%%%%%%%%%%%%
%                                                                      %
%     File: Thesis_Nomenclature.tex                                    %
%     Tex Master: Thesis.tex                                           %
%                                                                      %
%     Author: Andre C. Marta                                           %
%     Last modified : 21 Jan 2011                                      %
%                                                                      %
%%%%%%%%%%%%%%%%%%%%%%%%%%%%%%%%%%%%%%%%%%%%%%%%%%%%%%%%%%%%%%%%%%%%%%%%
%
% The definitions can be placed anywhere in the document body
% and their order is sorted by <symbol> automatically when
% calling makeindex in the makefile
%
% The \glossary command has the following syntax:
%
% \glossary{entry}
%
% The \nomenclature command has the following syntax:
%
% \nomenclature[<prefix>]{<symbol>}{<description>}
%
% where <prefix> is used for fine tuning the sort order,
% <symbol> is the symbol to be described, and <description> is
% the actual description.

% ----------------------------------------------------------------------
% Abbreviations [A]

\nomenclature[A]{GPU}{Graphics Processing Unit}
\nomenclature[A]{GPGPU}{General-purpose computing on Graphics Processing Units}

\nomenclature[A]{RSH}{Ray-Space Hierarchy}
