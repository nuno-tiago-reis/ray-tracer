%%%%%%%%%%%%%%%%%%%%%%%%%%%%%%%%%%%%%%%%%%%%%%%%%%%%%%%%%%%%%%%%%%%%%%%%
%                                                                      %
%     File: Thesis_Abstract.tex                                        %
%     Tex Master: Thesis.tex                                           %
%                                                                      %
%     Author: Andre C. Marta                                           %
%     Last modified :  2 Jul 2015                                      %
%                                                                      %
%%%%%%%%%%%%%%%%%%%%%%%%%%%%%%%%%%%%%%%%%%%%%%%%%%%%%%%%%%%%%%%%%%%%%%%%

\section*{Abstract}

% Add entry in the table of contents as section
\addcontentsline{toc}{section}{Abstract}

We present an algorithm for creating an n-level ray-space hierarchy (RSH) of coherent rays that runs on the GPU. Our algorithm uses rasterization to process the primary rays, then uses those results as the inputs for a RSH, that processes the secondary rays. The RSH algorithm generates bundles of rays; hashes them, according to their attributes; and sorts them. Thus we generate a ray list with adjacent coherent rays to improve the rendering performance of the RSH vs a more classical approach. In addition the scenes geometry is partitioned into a set of bounding spheres, intersected with the RSH, to further decrease the amount of false ray bundle-primitive intersection tests. We show that our technique notably reduces the amount of ray-primitive intersection tests, required to render an image. In particular it performs up to 50\% better in this metric than other algorithms in this class.

\vfill

\textbf{\Large Keywords:} {Rasterization, Ray-Tracing, Ray-Hashing, Ray-Sorting, Bounding-Cone, Bounding-Sphere, Hierarchies, GPU, GPGPU}

