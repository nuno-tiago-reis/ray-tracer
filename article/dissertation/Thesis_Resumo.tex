%%%%%%%%%%%%%%%%%%%%%%%%%%%%%%%%%%%%%%%%%%%%%%%%%%%%%%%%%%%%%%%%%%%%%%%%
%                                                                      %
%     File: Thesis_Resumo.tex                                          %
%     Tex Master: Thesis.tex                                           %
%                                                                      %
%     Author: Andre C. Marta                                           %
%     Last modified :  2 Jul 2015                                      %
%                                                                      %
%%%%%%%%%%%%%%%%%%%%%%%%%%%%%%%%%%%%%%%%%%%%%%%%%%%%%%%%%%%%%%%%%%%%%%%%

\section*{Resumo}

% Add entry in the table of contents as section
\addcontentsline{toc}{section}{Resumo}

Nós apresentamos um algoritmo para a criação de uma hierarquia multinível no espaço dos raios de raios coerentes que corre no GPU. O nosso algoritmo usa rasterização para processar os raios primários e posteriormente usa esses resultados como entrada para a criação da hierarquia de raios secundários. O algoritmo gera um conjunto de raios; cria índices para esses raios, de acordo com os seus atributos; e ordena-os. Deste modo geramos uma lista de raios que são coerentes com os seus adjacentes. Para melhorar o desempenho ainda mais, subdividimos a geometria da cena num conjunto de esferas envolventes que são posteriormente intersectadas com a hierarquia de raios para diminuir o número de falsos positivos nos testes de intersecção com primitivas. Demonstramos que a nossa técnica diminui de forma notável o número de intersecções entre primitivas necessárias para renderizar uma imagem, em particular até cerca de 50\% menos do que outros algoritmos nesta classe.

\vfill

\textbf{\Large Palavras-chave:} {Rasterização, Ray-Tracing, Indexaçao de Raios, Ordenamento de Raios, Cone Envolvente, Esfera Envolvente, Hierarquias, GPU, GPGPU}